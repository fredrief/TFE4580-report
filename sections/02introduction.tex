% !TEX root = ../report.tex
\chapter{Introduction}
Receiver systems with multiple inputs channels are becoming increasingly popular in a variety of applications. Multiple-input, multiple-output (MIMO) systems are being utilized in both radio communication, radar and acoustics. Medical ultrasound systems, in particular those designed for 3D imaging, might contain several thousand transducers in the probe. Common to all multi-input applications is the need for digitalizing a huge amount of analog input signals.

In most of the mentioned multi-input applications, this is done using general purpose ADCs, converting a single analog signal into a single digital bit stream. Depending on the application and the number of input channels, having one dedicated ADC per channel might not be feasible. For instance, in medical ultrasonic 3D imaging, a technique called sub-aperture processor (SAP) beamforming is used to limit the number of ADCs necessary in the system.

A system with tens, hundreds or even thousands of input channels motivates a rethinking of receiver system architecture. A structure consisting of a large number of general purpose ADCs, operating independently of each other, might not be the most efficient way of capturing the information from several analog input channels. If there exist some kind of prior knowledge about the input signal, this information could potentially be utilized for increased performance. Relevant prior knowledge could be the correlation between different channels or some statistical properties of an individual channel. Any prior knowledge will in principle imply that a general purpose ADC is redundant.

Utilizing this information is however not trivial. Interestingly, a newly developed concept called \textit{control-bounded conversion} enables novel ADC architectures that might address this issue in new and interesting ways.

\section{Control-Bounded Conversion}
Over the last few years, a group at the university ETH Zürich has developed a conceptually new approach to A/D conversion called control-bounded conversion \cite{cbc_2020_loeliger}. The main advantage of this approach is that it opens a new and wider design space, enabling ADC architectures that have previously been unimaginable. The latest contribution to control-bounded conversion is the doctoral thesis of Hampus Malmberg \cite{malmberg_thesis}. This thesis explores the design space associated with these converters, and proposes several different architectures suited for various applications.

One particularly interesting architecture is the so called \textit{Hadamard ADC}, which provides a new and interesting way of combining multiple input channels. Combining multiple input channels in one ADC enables the mentioned utilization of statistical information, yielding a potential performance increasement. The proposed architecture is however only evaluated on a theoretical level, and no optimized circuit implementation is reported so far.

\section{The Scope of the Thesis}
This thesis is an architecture study of control-bounded ADCs for multi-channel receiver applications. The presented work forms the background for a future circuit implementation, optimized for medical ultrasound imaging. The goal of the thesis is to study control-bounded ADCs for multi-channel systems in general, and the results will hopefully be relevant for other applications as well.

As the medical ultrasound application is of special interest, the  specifications listed in table \ref{tab:adc_specs} is kept in mind during the process. The specifications has been considered when choosing the frequency range for simulations, but is otherwise not limiting the relevance of the work.
% !TEX root = ../../report.tex

\begin{table}[htbp]
    \centering
    \caption{System Specifications}
      \begin{tabular}{lccr}
      \rowcolor[rgb]{ 0,  0,  0} \textcolor[rgb]{ 1,  1,  1}{\textbf{Parameter}}	 & \textcolor[rgb]{ 1,  1,  1}{\textbf{Symbol}}
                               & \textcolor[rgb]{ 1,  1,  1}{\textbf{Value}}       & \textcolor[rgb]{ 1,  1,  1}{\textbf{Comment}}    \\
      Carrier Frequency & $f_{ca}$ & \SI{5}{MHz} &                                                                                        \\
      Bandwidth & $\mathcal{B}$ & \SI{5}{MHz} & $2.5-7.5\si{MHz}$                                                                      \\
      Effective number of bits & ENOB & $>10$ bits &                                                                                    \\
      Signal to noise ratio & SNR & $>62$ dB &                                                                                           \\
      Technology & & $\SI{22}{\nano\metre}$ & FDSOI CMOS
      \end{tabular}
    \label{tab:adc_specs}
  \end{table}




The design space associated with control-bounded conversion is nearly infinite. Although the architectures evaluated in this thesis are rather simple, there are a lots of possible improvements to be considered. Some of them are pointed out, and many are probably not yet thought of. The challenge when working with control-bounded ADCs is not to come up with the good ideas, but to choose between them. The focus for this work, and the future circuit implementation, is to choose an architecture that is simple enough to allow for a reasonable implementation time, while at the same demonstrating some of the advanced features allowed by this new approach to A/D conversion.



\section{Main Contributions}
The main contribution of this thesis is the proposed ADC architecture, which is presented together with an analytic transfer function analysis, theoretical simulations and a discussion of some possible design challenges. This architecture will hopefully enable a more power efficient implementation of the Hadamard ADC introduced in \cite{malmberg_thesis}. Furthermore, the discussion of various design challenges provides a useful background when considering a control-bounded ADC for a multi-input application.

In addition, an important contribution of this work is to develop an understanding of the control-bounded ADCs operating principle, together with the development of necessary simulation framework. As this is a new and different approach to A/D conversion, the available literature on the topic is limited. Extensive use of simulations, together with calculations and hypothesis testing has been necessary to properly understand the concept. Together with the associated simulation framework, this constitutes an important basis for further work.






\section{Some Terminology}
Throughout the thesis, certain terms with slightly overlapping reach are frequently used. To avoid any misunderstanding, their usage is stated in this section.

\subsection{Architecture, Implementation, Topology and Design}
\textbf{Architecture} is used to describe the internal structure of a system. We use the term architecture on different abstraction levels, e.g. to describe the building blocks of the overall ADC or the internal building blocks of the analog system.

\textbf{Implementation} is reserved for the lowest abstraction level, i.e. transistor level implementations of involved components.

\textbf{Topology} is slightly overlapping with architecture. In this thesis, the word topology is placed between implementation and architecture on the abstraction level scale. We will for instance refer to different integrator topologies, which is the structure within an \enquote{interator-box}, but without considering any transistors.

\textbf{Design} is used either to describe the \textit{process} of creating something, or the complete result when this process is finished.


\subsection{Signal, Input, Channel, Element and State}
\textbf{Signal} is used to describe a time-varying, information-carrying quantity. A signal may be a scalar, e.g. $u(t) \in \R$, or a vector, e.g. $\utv \in \R^N$. We say that a multi-channel receiver system has one, multi-dimensional input signal.

An \textbf{Input} is a signal entering the system, or an internal subsystem of the receiver.

A \textbf{Channel} constitutes one dimension of the multi-dimensional interface to the analog world. Each channel carries one \textbf{element} of the input signal. An \textit{input channel} is in this thesis considered a redundant expression, and means the same as a channel.

A \textbf{State} is an element of the multi-dimensional state vector of the analog system. The analog system of the ADC is described using a state-space notation, and a state could for instance represent a voltage on an internal node of the analog system.






\section{Thesis Outline}
The thesis is structured as follows.

\textbf{Chapter \ref{sec:theory}} provides a short introduction to conventional oversampling A/D converters, and forms a basis for comparison with the presented control-bounded ADC.

\textbf{Chapter \ref{sec:cbadc}} gives a general, theoretical introduction to the control-bounded conversion principle. The goal of the chapter is to provide an understanding of the operating principle and the fundamental building blocks of the ADC. When describing the control-bounded ADC we follow the language and conventions established by previous publications, and the content of this chapter is very close to that of \cite{malmberg_thesis}, chapter 4. It is however less general and limited to what is considered relevant for the remaining part of the thesis.

\textbf{Chapter \ref{sec:ciadc}} presents the chain-of-integrator ADC, which is the simplest and most straight-forward control-bounded ADC architecture. This simple structure is the basis for the proposed ADC, and its analysis reveals useful results that is also applicable to other architectures.

\textbf{Chapter \ref{sec:HadamardADC}} covers the proposed ADC architecture. A discussion of possible hardware implementations and theoretical simulation results is presented. An analytic transfer function analysis is also provided.

\textbf{Chapter \ref{sec:final_discussion}} includes discussions of several topics that did not fit naturally in other parts of the thesis. Both potential for future improvement and limitations of the presented work is discussed.

Finally, \textbf{Chapter \ref{sec:conclusion}} concludes the thesis and present the current plan for future progress.

