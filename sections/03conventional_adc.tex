% !TEX root = ../report.tex

\chapter{Conventional A/D conversion}
\label{sec:conventional_adc}
Conventional A/D converters can generally be divided into two main categories; Nyquist rate- and oversampling converters.

\section{Nyquist Converters}
\label{subsec:nyquist_adc}
The classical approach to A/D conversion is to divide the conversion problem into two parts; sampling and quantization.
\begin{figure}[htbp]
    % !TEX root = ../../report.tex
\begin{tikzpicture}[circuit ee IEC]

\node[box] (lpf) at (0,0) {\lpf};
\node[box] (quantizer) at (3,0) {\quantizer};

\draw[arrow] (-1.5, 0) node[left]{$u(t)$} -- (lpf.west);
\draw (lpf.east)  node[right, yshift=-0.4cm]{$\tilde{u}(t)$} -- ++(0.5, 0) to[short, *-] ++(0.5, 0.2);
\draw[arrow] (1.5, 0) -- (quantizer.west);
\draw[arrow] (quantizer.east) -- ++(1, 0)  node[right]{$\hat{u}[k]$};


\end{tikzpicture}

    \centering
    \caption{The sample-centric view on A/D conversion.}
    \label{fig:sample_centric_adc}
\end{figure}

\section{Oversampling converters}
\label{subsec:oversampling_adc}
Robustness by introducing redundancy in the form of oversampling. Oversampling ration refers to the nyquist rate and defined by
\begin{equation}
    \text{OSR} = \frac{f_s}{2f_0}
\end{equation}

Straight-forward oversampling itself gives an advantage 3dB per doubling of OSR.

By introducing a feedback path with a loop filter, less error and better resolution can be obtained in a certain frequency band. Effect known as noise shaping. Such a system is known as a \sdmod. A discrete time \sdmod is shown in figure \ref{fig:dtsdmod}.

\begin{figure}[htbp]
    % !TEX root = ../../report.tex
\begin{tikzpicture}[]

    % --------------------- Nodes --------------------------------
    % LPF
    \node[box] (sampler) at (1.5,0) {S/H};
    \node[circle, draw, inner sep=0.5mm] (add) at (3,0) {\Large+};
    \node[box] (filter) at (4.5,0) {$G(z)$};
    \node[box] (quantizer) at (6.2,0) {\quantizer};
    \node[box] (decimation) at (8.5,0) {Dec};
    \node[dashed box, anchor=north, minimum width=5.3cm, minimum height=2.1cm] (sigma-delta) at (4.9,1.3) {};
    \node[above=5mm of sigma-delta.north, anchor=north]{$\Sigma\Delta$ modulator};

    % --------------------- Paths --------------------------------
    \draw[arrow] (0, 0) node[left]{$u(t)$} -- (sampler.west);
    \draw[arrow] (sampler.east) -- (add.west);
    \draw[arrow] (add.east) -- (filter.west);
    \draw[arrow] (filter.east) -- (quantizer.west);
    \draw[arrow] (quantizer.east) -- (decimation.west) node[left, yshift=-4mm, xshift=-4mm]{$s[k]$};
    \draw[arrow] (decimation.east) -- ++(0.5, 0) node[right]{$\hat{u}[k]$};

    \draw[arrow] (quantizer.east) ++(0.3, 0 ) to[short, *-] ++(0,1) -| (add.north) node[left, yshift=2mm]{\tiny $-$};


    \end{tikzpicture}

    \centering
    \caption{The sample-centric view on A/D conversion.}
    \label{fig:dtsdmod}
\end{figure}

A continuous time \sdmod is achieved by including the sampling in the feedback loop. In this case, the anti-aliasing filter is part of loop filter $G(e^{j\Omega})$. A continuous time \sdmod is shown in figure \ref{fig:ctsdmod}.

\begin{figure}[htbp]
    % !TEX root = ../../report.tex
\begin{tikzpicture}[]

    % --------------------- Nodes --------------------------------
    \node[circle, draw, inner sep=0.5mm] (add) at (1,0) {\Large+};
    \node[box] (filter) at (2.5,0) {$G(\omega)$};
    \node[box] (sampler) at (4.5,0) {S/H};

    \node[box] (quantizer) at (6.5,0) {\quantizer};
    \node[box] (decimation) at (8.5,0) {Dec};

    \node[box] (dac) at (4.5,1.5) {D/A};

    % --------------------- Paths --------------------------------
    \draw[arrow] (0, 0) node[left]{$u(t)$} -- (add.west);
    \draw[arrow] (add.east) -- (filter.west);
    \draw[arrow] (filter.east) -- (sampler.west);
    \draw[arrow] (sampler.east) -- (quantizer.west);

    \draw[*-, arrow] (quantizer.east) ++(0.3, 0 ) to[short, *-] ++(0,0.1) |- (dac.east);
    \draw[arrow] (dac.west) -| (add.north) node[left, yshift=2mm]{\tiny $-$};

    \draw[arrow] (quantizer.east) -- (decimation.west) node[left, yshift=-4mm, xshift=-1mm]{$s[k]$};
    \draw[arrow] (decimation.east) -- ++(0.5, 0) node[right]{$\hat{u}[k]$};

    \end{tikzpicture}

    \centering
    \caption{The sample-centric view on A/D conversion.}
    \label{fig:ctsdmod}
\end{figure}

\sdmod analysis


Transfer functions

