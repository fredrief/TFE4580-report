% !TEX root = ../report.tex

\chapter{Final Discussions}
\label{sec:final_discussion}

In this chapter provides some final discussions based on the results and experiences obtained from working with control-bounded ADC in general and the proposed Hadamard ADC architecture in particular. In this chapter, some discussions that did not fit naturally in the previous chapters are presented. There are also a lot of relevant topics that is beyond the scope of this thesis, and some of the limitations of the presented work is included.

\section{Additional Features of the Hadamard ADC}
\subsection{Shared Analog State Space}
It has been pointed out in section \ref{sec:advanced_DC} that there is a considerable potential in utilizing the shared control system of the Hadamard ADC. A shared control enables a more effective distribution of the available control resources, and hence a tighter state bound and increased performance.

In the multiple input Hadamard ADC the different input channels also shares the analog state-space. If there are an unequal distribution of the signal energy between the different input channels, combining them through the Hadamard matrix will have an averaging effect. More precisely, if the input signal vector is more concentrated in one of the physical dimensions than in one of the Hadamard dimensions, the inputs to the first column of integrators (i.e. $x_0$-$x_3$ of fig \ref{fig:HCI_AS_01}) will have a more even distribution of energy than the input channels itself. This effect could be utilized in terms of lowering the state bound or alternatively increasing the gain of the analog system, compared to the situation where each input channel is converted independently. Formally, the analog system could be scaled more towards $\E\left[ \norm{\utv}{2} \right]$, rather than $\E \left[ \norm{\utv}{\infty} \right]$, which would be required when each channel lies in a separate state-space.

This effect is illustrated in figure \ref{fig:signal_dimension}. The illustration is meant to give a simplified picture of the situation when a pulse is coming towards a receiver array at a steep angle. The receiver array has two elements, and due to the angle of incident, the pulse reaches the two elements at different times. The signal from the two elements are denoted $u_0(t)$ and $u_1(t)$ respectively. Through the Hadamard matrix, the input signals are combined in an orthogonal way. The resulting normalized outputs signals, denoted $\tilde{u}_0(t)$ and $\tilde{u}_1(t)$, are illustrated at the right side of the Hadamard matrix.

The pictured situation is the ideal application for a Hadamard ADC. At any time instance, the input signal lies almost entirely in one of the physical dimensions, i.e. $u_0(t)$ or $u_1(t)$. Hence, the signal energy is evenly distributed between the outputs $\tilde{u}_0(t)$ and $\tilde{u}_1(t)$. The magnitude of each output is reduced by a factor $\sqrt{2}$ relative to the input, but the effect is barely visible for the illustrated situation of two channels. For a larger system with hundreds or thousands of input channels, a scaling of $\sqrt{L}$ would be significant. As mentioned, this reduced signal strength could be utilized in terms of increased gain or tighter control bound. From equation (\ref{eq:analytic_snr_est}) we see that increasing the integrator gain, or lowering the state boundary, by a factor $\sqrt{L}$ would both give an SNR increasement proportional to $L$.
\begin{figure}[htbp]
    \centering
    \includegraphics[width=\linewidth]{figures/05hadamard/signal_dimension.pdf}
    \caption{An illustration of a pulse arriving at two different receiver elements at different times. $\utv = \Tr{\left( u_0(t) , u_1(t) \right)}$ is the input signal and $\tilde{\bm{u}}(t) = \Tr{\left( \tilde{u}_0(t) , \tilde{u}_1(t) \right)}$ is the output signal of the Hadamard matrix.}
    \label{fig:signal_dimension}
\end{figure}


Figure \ref{fig:hadamard_dimension} pictures the opposite situation. In this case the pulse arrives at the two inputs almost simultaneously, and the input signal lies almost entirely in the first Hadamard dimension. In consequence, almost all signal energy is concentrated at the first output of the Hadamard matrix, $\tilde{u}_0(t)$. In consequence, the magnitude of $\tilde{u}_0(t)$ is increased by $\sqrt{L}$ relative to the input, which would require a reduction in gain or state bound, and hence decreased SNR.
\begin{figure}[htbp]
    \centering
    \includegraphics[width=\linewidth]{figures/05hadamard/hadamard_dimension.pdf}
    \caption{An illustration of a pulse arriving at two different receiver elements at the same time. $\utv = \Tr{\left( u_0(t) , u_1(t) \right)}$ is the input signal and $\tilde{\bm{u}}(t) = \Tr{\left( \tilde{u}_0(t) , \tilde{u}_1(t) \right)}$ is the output signal of the Hadamard matrix.}
    \label{fig:hadamard_dimension}
\end{figure}

Any realistic situation would most likely be somewhere in between these two extremes. Whether combining the inputs in a Hadamard ADC gives a direct SNR improvement or not, will depend on which of the two extremes that is closest to the situation at hand. If we denote the output vector of the Hadamard matrix as $\tilde{\bm{u}}(t)$, the analog system will always have to be scaled towards $\E \left[ \norm{\tilde{\bm{u}}(t)}{\infty} \right]$. The question is wether or not $\E \left[ \norm{\tilde{\bm{u}}(t)}{\infty} \right] < \E \left[ \norm{\utv}{\infty} \right]$.






\subsection{Comparator Offset Voltage}
Another advantage of control-bounded ADCs that should be mentioned, is the sensitivity to offset voltage in the comparators. This applies to control-bounded converters in general and is not exclusive for the Hadamard ADC.

As highlighted several times, the only information about the control signals that is used by the digital estimator is that this was the signal needed to stabilize the internal states of the analog system. The digital estimator do not care about how these control signals relate to the input or the states of the analog system. Note that $\Gmattilde$ is not used by the DE in figure \ref{fig:de_problem}. This means that the output estimate is completely invariant to the offset voltage of the comparators, as long as the offset voltage is not disabling the digital control from bounding the state vector.
\textcolor{red}{Is this interesting?}





\subsection{Noise in the Hadamard System}
An important non-ideality that is not treated in this thesis is the noise generated by the components of the analog system. When the signal is integrated in a chain structure, the first integrator will be the most critical and most of the power budget will be consumed by this integrator in order to keep its noise contribution at a minimum. In the Hadamard ADC however, all integrators in the first column will contribute equally to the total noise \textcolor{red}{budget(?)}. It was shown in \cite{malmberg_thesis} that it is possible tune the situation by amplifying the different signal dimensions unequally. By increasing the gain of one/some signal dimensions, and reducing it on others, the noise contribution is distributed unequally between the integrators. This however comes at an expense of reduced nominal performance.

The input node of the system is critical, as this is where the information carrying input signal is at its weakest. The vulnerable input signal should be handled with care, and connecting more active components to this node does not intuitively sound beneficial. However, when designing the active components, trade-offs will have to be made between noise, speed, linearity, etc., and other design specification might influence the cost of reaching a certain noise requirement. We therefore argue that evaluating the performance of each architecture in terms of noise suppression should be done while considering transistor level implementations of the involved components.













\section{Limitations of the Presented Work}
Some limitations of the presented work is discussed in this section.


\subsection{Filter Complexity}
The work of this thesis is mainly concerned with the analog system and the digital control of the proposed architecture. The simulations are performed using an offline estimation filter implemented in Python. An efficient, online filter algorithm is presented in \cite{malmberg_thesis}, but is not tested in this work. The power consumed by the estimation filter will of course count on the total power budget of the ADC, and should therefore be taken into account when evaluating the overall system performance.

For larger systems with several input channels, it might be more efficient to run the algorithm on a micro-controller, rather than using a dedicated HDL implementation of the filter algorithm. These are interesting questions that remains for future work.


\subsection{Complex Poles and Optimized Transfer Function}
In this thesis, the analog transfer function of both the chain-of-integrators and the Hadamard system constitutes a chain of first order integrations. However, in a practical realization it would be favorable to have more advanced transfer functions in the analog system. Adding zeros in the transfer function enables band-pass and notch filter realizations, and complex pole pairs enables sharper transitions between pass-band and stop-band. This way, more of the analog system gain may be concentrated in the frequency band of interest, resulting in an overall performance increasement. Optimizing the noise transfer function is an important part of the design of $\sd$ modulators and should be done in control-bounded ADCs as well.

It was shown in \cite{malmberg_thesis} how a chain-of-integrator-like structure can be modified to allow the design of any general $N$-th order transfer function polynomial. Poles are introduces to the transfer function by connecting the different states through additional feedback loops, and zeros by feed-forward paths from the input. The same general transfer function is also achievable for the Hadamard system by using a similar structure.

Evaluating the proposed architecture with an optimized transfer function is unfortunately beyond the scope of this paper. The main focus has been to find and evaluate an efficient realization of the Hadamard analog system, and optimized transfer functions is kept in mind as an important part of future development.



\subsection{Quantitative Measures}
The simulations presented in this thesis are only meant to give a qualitative indication of the performance of the different architectures. When designing for a specific application one would want to find the optimum filter order, OSR, integrator gain, state boundary etc., in order to reach the performance requirements with the lowest possible power consumption.

The simulations performed in this thesis are highly ideal, and no attempt has been made to do such optimizations. These optimizations will be considered together with transistor level implementations in the given technology and the goal of this thesis is to provide a useful background for the next part of the design process.



